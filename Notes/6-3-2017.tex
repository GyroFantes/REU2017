\documentclass{article}
\usepackage{format}
\title{Generic Notes}
\author{Forest Yang}
\date{June 5, 2017}


\begin{document}
\maketitle
\section{Introduction to the propagator} 
Consider the classical mechanics problem of two blocks of mass $m$ in a well coupled by springs to the walls and a spring between them (three springs in total). Denoting $x_1$ and $x_2$ as the displacement of each block from the equilibrium position, (note there is a unique equilibrium configuration of the masses, since we can set up two equations with two unknowns and solve them) Hooke's law implies that
\begin{align*}
\begin{aligned}
& \ddot{x}_1 & = && -2\frac{k}{m}x_1 + \frac{k}{m}x_2 \\
& \ddot{x}_2 & = && \frac{k}{m}x_1 - 2\frac{k}{m}x_2  
\end{aligned}
\end{align*}
Written in matrix form, this is 
\[
\begin{bmatrix} \ddot{x}_1 \\ \ddot{x}_2 \end{bmatrix} = \begin{bmatrix} \Omega_{11} & \Omega_{12} \\ \Omega_{21} & \Omega_{22} \end{bmatrix} 
\begin{bmatrix} x_1 \\ x_2 \end{bmatrix} \qquad \begin{array}{ll} \Omega_{11} = -\frac{2k}{m} & \Omega_{12} = \frac{k}{m} \\ \Omega_{21} = \frac{k}{m} & \Omega_{22} = -\frac{2k}{m} \end{array}
\]
From which it is apparent that the matrix $\Omega$ is hermitian. This can be represented in a more abstract form as \begin{equation*} \ket{\ddot{x}(t)} = \Omega\ket{x(t)} \end{equation*}
In this representation, the equivalent of $\begin{bmatrix} 1 \\ 0 \end{bmatrix}$ is $\ket{1}$ , representing displacement of the first mass from its equilibrium position and the equivalent of $\begin{bmatrix} 0 \\ 1 \end{bmatrix}$ is $\ket{2}$, representing displacement of the second mass from its equilibrium position. Therefore \begin{equation*} \ket{x(t)} = x_1\ket{1} + x_2 \ket{2} \end{equation*}
The abstract vector equation can be put into a form that more closely relates to the original matrix equation by projecting onto $\ket{1}$ and $\ket{2}$, and using the completeness relation $\ket{1}\bra{1} + \ket{2}\bra{2} = I$:
\begin{align*}
\begin{aligned}
 \ddot{x}_1 & = & \braket{1 | \ddot{x}(t)} && = && \bra{1} \Omega (\ket{1}\bra{1} + \ket{2}\bra{2}) \ket{x(t)} && = && \braket{1|\Omega|1}\braket{1|x(t)} + \braket{1|\Omega|2}\braket{2|x(t)} && = &&\Omega_{11}x_1 + \Omega_{12}x_2 \\
\ddot{x}_2 & = & \braket{2 | \ddot{x}(t)} && = && \bra{2} \Omega(\ket{1}\bra{1} + \ket{2}\bra{2}) \ket{x(t)} &&= && \braket{2 \Omega | 1}\braket{1|x(t)} + \braket{2|\Omega|2}\braket{2|x(t)} && = &&
\Omega_{21}x_1 + \Omega_{22}x_2
\end{aligned}
\end{align*} 
To be continued.
\section{Green's Function, Equation of Motion}
The Schr\"{o}dinger picture considers the wavefunction time-dependent and the operator time-independent. On the other hand, the Heisenberg picture considers the wavefunction time-independent and the operator time-dependent. These viewpoints are mathematically equivalent, because the Heisenberg time-dependent operator $\hat{o}_{\mathcal{H}}$ is defined in terms of the Schr\"odinger time independent operator $\hat{o}_{\mathcal{S}}$ as 
\begin{equation*} \hat{o}_{\mathcal{H}} = e^{i\hat{H}t}\hat{o}_{\mathcal{S}}e^{-i\hat{H}t}\end{equation*}
so that the evolution of the expectation of an operator remains the same, as seen in 
\begin{equation*} \braket{\Psi(t)|\hat{o}_\mathcal{S}|\Psi(t)} = \braket{e^{-i\hat{H}t}\psi_0|\hat{o}_\mathcal{S}|e^{-i\hat{H}t}\psi_0} = \braket{\psi_0|e^{i\hat{H}t}\hat{o}_\mathcal{S}e^{-i\hat{H}t}|\psi_0} = \braket{\psi_0|\hat{o}_\mathcal{H}|\psi_0}\end{equation*}
Heisenberg's equation can be derived using the chain rule as follows:
\begin{equation*} \frac{\partial \hat{o}_\mathcal{H}}{\partial t} = (i\hat{H})e^{i\hat{H}t}\hat{o}_\mathcal{S}e^{-i\hat{H}t} + e^{i\hat{H}t}\hat{o}_\mathcal{S}e^{-i\hat{H}t}(-i\hat{H}) = i[\hat{H}, \hat{o}_\mathcal{H}]\end{equation*} 
This equation may appear without the factor of $i$ attached to the commutator when working with imaginary time. \\
A useful property is that 
\begin{equation*}[\hat{a}_\mathcal{H}, \hat{b}_\mathcal{H}] = [\hat{a}_\mathcal{S}, \hat{b}_\mathcal{S}]_\mathcal{H} \end{equation*}
Similarly 
\begin{equation*}\{\hat{a}_\mathcal{H}, \hat{b}_\mathcal{H}] = \{\hat{a}_\mathcal{S}, \hat{b}_\mathcal{S}\}_\mathcal{H} \end{equation*}
The Green's function can be approached using the framework of equation of motion. As a reminder, the Green's function was 
\begin{equation*}G(\tau, \tau') = -\frac{1}{z}\tr\big[e^{-\beta\hat{H}}T_\tau C(\tau)C^\dagger(\tau')\big] = -\ex{T_\tau C(\tau)C^\dagger(\tau')} \qquad \ex{\hat{o}} := \frac{1}{z}\tr e^{-\beta\hat{h}}\hat{o}   \end{equation*}
For now, as a notation that may come in handy in the future, let the subscripts $ij$ in $G_{ij}(\tau,\tau')$ denote subscripts of the $C(\tau)$ and $C^\dagger(\tau')$ respectively. Also, for the purpose of taking a derivative, it's helpful to rewrite the function as 
\begin{equation*} G_{ij}(\tau, \tau') = -\hv(\tau-\tau')\ex{C(\tau)C^\dagger(\tau')} + \hv(\tau'-\tau)\ex{C^\dagger(\tau')C(\tau)} \end{equation*} 
Since the derivative of the heaviside function is the delta function. Then (using time-translational invariance and an implicit 0 for the second argument of Green's function)  
\begin{align*}
\frac{\partial G_{ij}(\tau)}{\partial \tau} &= -\delta(\tau)\ex{C_i(\tau)C_j^\dagger(0) + C_j^\dagger(0) C_i(\tau)} -\hv(\tau)\ex{[\hat{H}, C_i(\tau)]C_j^\dagger(0)} + \hv(-\tau)\ex{C_j^\dagger(0),[\hat{H}, C_i(\tau)]} \\
&= -\delta(\tau)\delta_{ij} - \hv(\tau)\ex{\mu C_i(\tau)C_j^\dagger(0)} + \theta(-\tau)\ex{C_j^\dagger(0), \mu C_i(\tau)} \\
&= -\delta(\tau)\delta_{ij} + \mu G_{ij}(\tau) 
\end{align*}
Rearranging this:
\begin{equation*}
\implies (\mu - \frac{\partial}{\partial \tau})G_{ij} = \delta(\tau)\delta_{ij} \end{equation*}
If the right side is interpreted to be ``unity," then one might interpret the left side as $``G^{-1} G" $. Now, what happens when the Hamiltonian is switched from $-\mu c^\dagger c$ to, say, \begin{equation*}-\mu (c_\uparrow^\dagger c_\uparrow + c_\downarrow^\dagger c_\downarrow) + U\hat{n}_\uparrow\hat{n}_\downarrow \end{equation*} The only thing that changes in the above derivation is $[\hat{H}, C(\tau)]$. Recalling some commutator algebra with creation/annihilation operators:
\begin{align*}
[\hat{n}_{i\sigma}, c_{j\sigma'}] &= c^\dagger_{i\sigma}\{c_{i\sigma}, c_{j\sigma'}\} - \{c^\dagger_{i\sigma}, c_{j\sigma}\}c_{i\sigma} \\&= -\delta_{ij}\delta_{\sigma\sigma'}c_{i\sigma} \\
[\hat{n}_{i\uparrow}\hat{n}_{i\downarrow}, c_{j\sigma}] &= \hat{n}_{i\uparrow}[\hat{n}_{i\downarrow}, c_{j\sigma}] + [\hat{n}_{i\uparrow}, c_{j\sigma}]\hat{n}_{i\downarrow} \\
&= -\delta_{ij}(\delta_{\sigma\downarrow}\hat{n}_{i\uparrow}c_{i\downarrow} + \delta_{\sigma\uparrow}\hat{n}_{i\downarrow}c_{i\uparrow})
\end{align*}
Therefore, the final equation of motion doesn't change much. Concluding the derivation
\begin{align*}
\frac{\partial G_{\sigma\sigma'}}{\partial \tau} &= -\delta(\tau) -\hv(\tau)\ex{[\hat{H}, C_\sigma(\tau)]C_{\sigma'}^\dagger(0)} + \hv(-\tau)\ex{C_{\sigma'}^\dagger(0),[\hat{H}, C_\sigma(\tau)]}  \\
&= -\delta(\tau) - \hv(\tau)\ex{(\mu C_\sigma(\tau) - U\hat{N}_{\bar{\sigma}}C_\sigma(\tau))C_{\sigma'}^\dagger(0)} + \hv(\tau)\ex{C_{\sigma'}^\dagger(0) (\mu C_\sigma(\tau) - U\hat{N}_{\bar{\sigma}}C_\sigma(\tau))} \\
&= -\delta(\tau) +\mu G_{\sigma\sigma'} + U\hv(\tau)\ex{\hat{N}_{\bar{\sigma}}C_\sigma(\tau)C_{\sigma'}^\dagger(0)} - U\hv(\tau)\ex{C_{\sigma'}^\dagger(0)\hat{N}_{\bar{\sigma}}C_\sigma(\tau)} \\
&= -\delta(\tau) + (\mu - U\hat{N}_{\bar{\sigma}}) G_{\sigma\sigma'} 
\end{align*}
$\hat{N}$ is the number operator so it can be moved out of the expectation freely (? prove this later). Therefore the obtained equation of motion is 
\begin{equation*} (\mu - U\hat{N}_{\bar{\sigma}} - \frac{\partial}{\partial \tau})G_{\sigma\sigma'} = \delta(\tau) \end{equation*}
\subsection{Action}
What if you introduce a disturbance to your system over a period of time? Say the hamiltonian is \begin{equation*} \hat{H} = -\mu c^\dagger c \end{equation*} and the disturbance is modeled by 
\begin{equation*} S(U) = e^{-i \int \hv(t)\hv(t_0 - t) U c^\dagger(t) c(t) dt} = e^{-i \int_0^{t_0}U c^\dagger (t) c(t) dt} \end{equation*} 
which changes the Green's function (this one is the real time version) to 
\begin{equation*}G(t, t') = -i\ex{T_t C(t)C^\dagger(t') S(U)} \end{equation*}
The presence of the $S(U)$ modifies the value of the normalization constant $z$ (so that $S(U)$ really defines a new expectation) to 
\begin{align*}
z(u) &= \tr \big[ e^{-i\int_0^{t_0} Uc^\dagger c dt} e^{\mu\beta c^\dagger c} \big] \\
&= 1 + e^{\mu\beta} e^{-i\int_0^{t_0} U dt} \\
&= 1 + e^{\mu\beta - iUt_0}
\end{align*}
Now let's compute this Green's function for a specific order of $0, t_0, t, t'$. There are essentially 12 orders. The order we will compute Green's function for is $t < 0 < t' < t_0$. The first step is to find an expression for $C(t) = e^{i\hat{H}t}c e^{-i\hat{H}t}$ and one for $C^\dagger(t') = e^{i\hat{H}t'}c^\dagger e^{-i\hat{H}t'}$. Using Heisenberg's equation:
\begin{align*}
\frac{\partial C(t)}{\partial t} &= i[-\mu c^\dagger c, C(t)] \\
&= i \mu C(t) \\
\Rightarrow C(t) &= e^{i\mu t} c \\
\frac{\partial C^\dagger(t')}{\partial t'} &= i[-\mu c^\dagger c, C^\dagger(t')] \\
&= -i \mu C^\dagger(t') \\
\Rightarrow C^\dagger(t') &= e^{-i\mu t} c^\dagger
\end{align*}
Evaluating Green's function supposing $t < 0 < t' < t_0$:
\begin{align*}
G(t,t') &= -i\hv(t' > t_{int} > t) \ex{C^\dagger(t') e^{-i\int_0^{t'} U c^\dagger c\, dt} C(t)} + i\hv(t_{int} > t' > t)   \ex{e^{-i\int_{t'}^{t_0} U c^\dagger c\, dt} C^\dagger(t') C(t)} \\
&= -\frac{i}{z} \tr \big[e^{-\beta\hat{H}} e^{-i\mu t'} c^\dagger e^{-i\int_0^{t'} U c^\dagger c \, dt} e^{i\mu t} c \big] + \frac{i}{z} \tr \big[ e^{-\beta\hat{H}} e^{-i\int_{t'}^{t_0} U c^\dagger c\, dt} e^{-i\mu t'} c^\dagger e^{i\mu t} c \big] \\
&= -\frac{i e^{i\mu(t-t')}}{z}\big[\braket{0|c^\dagger e^{-i\int_{0}^{t'} U c^\dagger c\,dt} c |0} + e^{\beta\mu} \braket{1| e^\dagger e^{-i\int_0^{t'} Uc^\dagger c\, dt} c | 1} - e^{\beta\mu - iU(t_0-t')}\big] \\
&= -\frac{ie^{i\mu(t-t')}}{z}(0+ e^{\beta\mu}(1) - e^{\beta\mu - iU(t_0 - t')}) \\
&= -\frac{ie^{\mu(\beta + i(t-t'))}}{z}(1- e^{-iU(t_0-t')}) 
\end{align*}
\end{document}